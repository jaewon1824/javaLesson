복습 QnA

1.메소드의 정의 - 왜 만드는 것일까요?       (다른 프로그래밍언어에서 '함수')
                자주 사용되는 기능을 main 에서 분리시킨다.
                코드를 여러 번 작성하지 않고 재사용.
2.메소드의 인자, 매개변수란 무엇인가요?     메소드 실행한 결과값.
            인자 argument : 둘 사이에 값을 연결시킨다.
                        메소드 호출 시 입력값을 메소드 정의한 실행 부분으로 연결시켜준다.
            매개변수 : 선언된 변수 () 안에서 선언된 변수.
        결론은 인자와 매개변수는 같은 의미입니다.
3.메소드의 리턴은 무엇인가요?       
4.static 메소드의 특징은 무엇인가요.
        답 : 클래스 이름으로 직접 실행(호출)
        예시 : Character.isUpperCase(c) String.format ("%d",val)     Inerger.valueOf
(비교) 인스턴스 메소드는 객체를 만들어서 메소드를 호출
(인스턴스 메소드 예시) message.length() , sc.nextInt() ,
**정리 : 메소드는 인스턴스 메소드와 static 메소드로 분류할 수 있습니다.     